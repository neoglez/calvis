\documentclass[10pt,twocolumn,letterpaper]{article}

\usepackage{iccv}
\usepackage{times}
\usepackage{epsfig}
\usepackage{graphicx}
\usepackage{amsmath}
\usepackage{amssymb}

% Include other packages here, before hyperref.

% If you comment hyperref and then uncomment it, you should delete
% egpaper.aux before re-running latex.  (Or just hit 'q' on the first latex
% run, let it finish, and you should be clear).
\usepackage[pagebackref=true,breaklinks=true,letterpaper=true,colorlinks,bookmarks=false]{hyperref}

% \iccvfinalcopy % *** Uncomment this line for the final submission

\def\iccvPaperID{****} % *** Enter the ICCV Paper ID here
\def\httilde{\mbox{\tt\raisebox{-.5ex}{\symbol{126}}}}

% Pages are numbered in submission mode, and unnumbered in camera-ready
\ificcvfinal\pagestyle{empty}\fi

\begin{document}

%%%%%%%%% TITLE
\title{CALVIS: Chest, wAist and peLVIS circumference from 3D human body meshes 
as ground truth for deep learning}

\author{Yansel González Tejeda\\
Paris Lodron University of Salzburg\\
{\tt\small yansel.gonzalez-tejeda@stud.sbg.ac.at}
% For a paper whose authors are all at the same institution,
% omit the following lines up until the closing ``}''.
% Additional authors and addresses can be added with ``\and'',
% just like the second author.
% To save space, use either the email address or home page, not both
\and
Helmut Mayer\\
{\tt\small helmut@cosy.sbg.ac.at}
}

\maketitle
% Remove page # from the first page of camera-ready.
\ificcvfinal\thispagestyle{empty}\fi

%%%%%%%%% ABSTRACT
\begin{abstract}
   In this paper we present CALVIS, a method to calculate chest, waist and 
   pelvis circumference from 3D human body meshes. The idea is to use these 
   human body dimensions as ground truth for deep learning. Previous work had 
   used the large scale CAESAR dataset, calculated $\textit{manually}$ these 
   anthropometrical measurements or directly acquired them from a person. The 
   problem is that acquiring these data is a cost and time consuming endeavor. 
   Hence, instead of measuring real people, we employ a generative model of 
   humans, which is able to generate any number of humans with different shapes 
   and poses from a single human template. We then compute chest, waist and 
   pelvis circumference from these models (represented as a 3D mesh) to 
   balabla. We conduct two experiments. In the first experiment we synthesize 
   10 human body meshes. Then we apply CALVIS to calculate chest, waist and 
   pelvis circumference. We evaluate the results qualitatively. We observe that 
   the measurements can indeed be used to estimate the shape of a person. The 
   second experiment serves as a proof-of-concept where we use the calculated 
   human dimensions as a ground truth to train an artificial neural network. 
   The idea is to establish the plausibility of our approach. After having 
   trained the network with our 
   data, we proof that the network is able to conduct this task.
\end{abstract}

%%%%%%%%% BODY TEXT
\section{Introduction}

The general idea is to train a prediction model (in our case a {\em 
convolutional neural network} (CNN)) with examples of images of humans whose 
body measurements are known, and use the trained model as a predictor of body 
dimensions given arbitrary images of humans. The main problem with this 
approach is a sufficient number of available training data, as acquiring these 
data is a cost and time consuming endeavor. Hence, instead of measuring real 
people, we employ a generative model of humans, which is able to generate any 
number of humans with different shapes and poses from a single human template. 
We then compute the body dimensions algorithmically from these models 
(represented as a 3D mesh) to be used as ground truth for synthetic images of 
the model.

We conduct two experiments. In the first experiment we synthesize 10 human body meshes. Then we apply our method to calculate chest, waist and pelvis circumference. We evaluate the results qualitatively. We observe that the measurements can indeed be used to estimate the shape of a person. The second experiment serves as a proof-of-concept where we input the calculated human dimensions to an artificial neural network. The idea is to establish the plausibility of our approach. After having trained the network with our data, we proof that the network is able to conduct this task.

Problem statement: given a 3D human body mesh $\mathcal{M}$ we look for a 
method capable to automatically output chest, waist and pelvis circumference.

%------------------------------------------------------------------------
\section{Approach}
In this work we synthesize 3D human meshes using the SMPL
model \cite{Loper.2015}. This model is at its top level a skinned articulated 
model, i.e., 
consists of a 
surface mesh that mimics the skin and a skeleton related to that mesh. It is 
defined by a mean 
template shape represented by a vector of $N$ concatenated vertices 
$\mathbf{\bar{T}}$ in the zero pose, $\vec{\theta}$. In order to synthesize a 
new human mesh one has to deform the provided template mesh by 
setting shape parameters $\vec{\beta}$ and pose parameters $\vec{\theta}$. The 
model provides learned parameters
\begin{equation} \label{eq:smpl_params}
\Phi = \{\mathbf{\bar{T}}, \mathcal{W}, \mathcal{S}, \mathcal{J}, 
\mathcal{P}\}
\end{equation}
As mentioned above $\mathbf{\bar{T}}$ is the mean template shape. The weight 
matrix $\mathcal{W}$ represents how much the rotation of skeleton parts affects 
the vertices. In addition, the matrices $\mathcal{S}$ and $\mathcal{P}$ define 
linear functions that are used to deform $\mathbf{\bar{T}}$ and the matrix 
$\mathcal{J}$ predicts skeleton rest joint locations from vertices in the rest 
pose. We held fix these parameters during the synthesis.

Let us consider a human body mesh $\mathcal{M}$. Our method requires that 
$\mathcal{M}$ is standing with arms raised 
parallel to the 
ground at shoulder height at a $90^\circ$ angle. In the line 
of previous work (\cite{Dibra.2016a})), we name this pose the zero (also 
normalized (rf)) pose $\vec{\theta}^*$. Additionally, we assume that the mesh 
has LSA orientation, e.g., x, y and z axis are positively directed from 
right-to-left, inferior-to-superior and posterior-to-anterior, respectively. If 
the mesh has another orientation we can always apply transformations to orient 
it LSA.

Intuitively, we would like to measure the chest circumference bellow the arms 
at the widest part of the torso and the waist circumference at the 
narrowest part after the chest but above the hips. Similarly, the hip 
circumference is measured often around the widest part of hips and buttocks. We 
can formalize this intuition by considering the cross-sectional length of the 
2D curve along the y-axis. Since $\mathcal{M}$ is triangulated, we can use a 
plane $\boldsymbol{\pi}$ to intersect the mesh. The boundary of this 
intersection is a collection of connected segments $s_i$. Therefore, we can 
determine this boundary length as
\begin{align}
BL = \sum_{i = 1}^{i = n}s_i
\end{align}  

Next, starting from the bounding box top part we cut the mesh into evenly 
spaced slices $m$ and compute the total sum. 

%------------------------------------------------------------------------
\section{Experiments and Results}


%------------------------------------------------------------------------
\section{Conclusion}

You must include your signed IEEE copyright release form when you submit
your finished paper. We MUST have this form before your paper can be
published in the proceedings.

{\small
\bibliographystyle{ieee}
\bibliography{egbib}
}

\end{document}
